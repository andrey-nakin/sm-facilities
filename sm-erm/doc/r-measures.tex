\label{sec_r_measures}

Сопротивление Образца измеряется 4-х контактным методом, для чего к Образцу подводятся два потенциальных и два токовых контакта. Сопротивление определяется одним из следующих способов:

\subsubsection{Вольтметром/ амперметром}

Используются вольтметр, амперметр и ИП. Амперметр и ИП включены последовательно с Образцом, амперметр подключён к токовым контактам Образца. Вольтметр включён параллельно с Образцом и подключён к его потенциальным контактам. Сопротивление Образца определяется как $R = V/I$, где $R$~--- искомое сопротивление, $V$~--- показания вольтметра, $I$~--- показания амперметра.

Данный способ наиболее универсальный и точный, но требует максимального количества задействованных мультиметров.

\subsubsection{Омметром}

Используется омметр, подключённый к Образцу 4-х контактным способом. Сопротивление Образца определяется непосредственным считыванием показаний омметра.

Данный способ наиболее простой, но не годится в том случае, когда сопротивление Образца слишком мало или велико, то есть выходит за диапазон точных измерений омметра.

\bigskip 

Во всех вышеприведённых способах функции вольметра, амперметра или омметра выполняет мультметр, переведённый в соответствующий режим.
