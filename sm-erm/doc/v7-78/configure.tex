В поле \CTL{Адрес} вводится или выбирается из списка значений адрес устройства. Как правило, если устройство подключено к ПК и должным образом обнаружено библиотекой VISA, его адрес должен быть в списке.

Поле \CTL{Число циклов 50 Гц на измерение} определяет точность измерений. Мультиметр определяет измеряемую величину не одномоментно, а путём многократных измерений в течении некоторого времени. Если это время кратно продолжительности периода питания сети 220~В, то влияние биений, связанных с переменным током питания, будет сведено к минимуму. Время каждого измерения определяется по формуле:

\begin{equation}
t = N_{50} \cdot 20 \cdot 2,
\end{equation}

\noindent где $t$~--- время в мс, $N_{50}$~--- число циклов сети питания 220~В, $20$~--- продолжительность одного периода при частоте 50~Гц, $2$~--- коэффициент вызванный тем, что каждое измерение фактически производится дважды: первый раз оно производится для измерения <<нуля>>. Такой способ измерений предотвращает <<уход нуля>> при продолжительных измерениях. Таким образом, при $N_{50} = 1$ продолжительность одного измерения равна $40$~мс, а при $N_{50} = 100$ --- $4$~с.

Рекомендуемое значение параметра~--- 10, в этом случае обеспечивается точность до \mbox{6 1/2} значащих десятичных знаков. Не рекомендуется выбирать в качестве числа циклов дробное значение, поскольку в данном случае результаты измерений будут содержать заметный шум на частоте $50$~Гц. 

Подробнее см. в документации к мультиметру в разделах <<Подавление помех сети электропитания>> и <<Автоматическая установка нуля>>.

Для проверки правильности параметров можно установить пробное соединение с устройством, для этого нажмите кнопку \CTL{Опрос}. Программа попытается установить связь, после чего сообщит о результатах. Не рекомендуется производить опрос в процессе измерений, это может привести к потере данных.
