Поскольку данное устройство подключено не к ПЭВМ, а к АС-4, сперва необходимо указать в поле \CTL{Порт для АС-4} имя соответствующего порта.

Далее нужно в поле \CTL{Сетевой адрес МВУ-110} указать адрес устройства сети RS-485, управляемой АС-4. Адресом является целое число в диапазоне 0--248, кратное 8.

Адрес может быть подписан на самом устройстве, в противном случае его можно узнать при помощи программы <<Конфигуратор МВУ-110>>, идущей в комплекте с устройством. Подробную информацию см. в документации к МВУ-110 и программе-конфигуратору.

\IMPORTANT{После работы с конфигуратором устройство МВУ-110 переходит в режим работы с протоколом OWEN. Для возврата на протокол ModBus, поддерживаемый Программой, нужно на несколько секунд обесточить устройство.}

Для проверки правильности параметров можно установить пробное соединение с устройством, для этого нажмите кнопку \CTL{Опрос}. Программа попытается установить связь, после чего сообщит о результатах. Не рекомендуется производить опрос в процессе измерений во избежание потери данных.